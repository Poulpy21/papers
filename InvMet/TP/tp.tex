\documentclass[11pt,a4paper]{article} 

\usepackage[T1]{fontenc}
\usepackage[utf8]{inputenc}
\usepackage{lmodern}
\usepackage[french,english]{babel}

\usepackage{amsthm}
\usepackage{float}
\usepackage{lmodern}%pour un meilleur rendu des polices
\usepackage{verbatim}%du texte non interprt
\usepackage[cmex10]{amsmath}
\usepackage{amssymb}%maths
\usepackage{xspace}
\usepackage[dvipsnames,svgnames,table]{xcolor}
\usepackage{listings}
\usepackage{fancyhdr}
\usepackage{etoolbox}
\usepackage{titlesec}
\usepackage{titletoc}
\usepackage{lastpage}
\usepackage[bookmarks=true,bookmarksnumbered=true]{hyperref}
\usepackage{ctable} % for \specialrule command
\usepackage{cite}
\usepackage{algorithm2e}
\usepackage{alltt}
\usepackage{array}
\usepackage{mdwmath}
\usepackage{mdwtab}
\usepackage{eqparbox}
\usepackage[caption=false,font=normalsize,labelfont=sf,textfont=sf]{subfig}
\usepackage{dblfloatfix}
\usepackage{url}
\usepackage{tipa}
\usepackage{stmaryrd}
\usepackage{mathrsfs}

%\usepackage{natbib}
%\usepackage[pdftex]{graphicx}
%\usepackage{framed}
%\usepackage[usenames]{color}

\graphicspath{{img/}}
\DeclareGraphicsExtensions{.pdf,.jpeg,.jpg,.png}

%% taille du papier
\textwidth 16 true cm
\textheight 24 true cm
\addtolength{\hoffset}{-1.5cm}
\addtolength{\voffset}{-1.5cm}

%-------- couleurs
\definecolor{grisf}{rgb}{.47,.47,.47} % barre de droite gris fonce
\definecolor{imag}{RGB}{50,0,100}
\definecolor{darkimag}{RGB}{65,15,100}
\definecolor{darkgreen}{RGB}{65,15,100}
\newcommand{\colorc}{\color{darkimag}}
\newcommand{\colorb}{\color{darkimag}}
\newcommand{\colora}{\color{Blue}}

%----------- sections et TOC
% chapitres
\titleformat{\chapter}[display]
  {\normalfont\sffamily\bfseries\huge\colora\centering}{\thechapter}{1ex}
  {{\titlerule[1pt]}\vspace{1.3ex}}[\vspace{1ex}{{\titlerule[1pt]}}]
  
% chapitres etoiles  
\titleformat{name=\chapter,numberless}[display]
  {\normalfont\sffamily\bfseries\LARGE\colora\centering}{}{1ex}
  {{\titlerule[1pt]}\vspace{1.3ex}}[\vspace{1ex}{\titlerule[1pt]}\vspace{2ex}]
  
% sections  
\titleformat{\section}[hang]{\Large\normalfont\sffamily\bfseries\colora}{{\thesection\, }}{0 em}
  {}[{\titlerule[1pt]}\vspace{1ex}]

  
% sous section, sous sous sec, paragraphes  
\titleformat{\subsection}[hang]{\Large\normalfont\sffamily\bfseries\colorc}{{\thesubsection\, }}{0 em}
  {}[{\titlerule}\vspace{.7ex}]
\titleformat{\subsubsection}[hang]{\normalfont\sffamily\bfseries\large}{{\thesubsubsection\, }}{0 em}
  {}[{\color{grisf}\titlerule}\vspace{3pt}]
\titleformat{\paragraph}[runin]{\normalfont\sffamily\bfseries\colorb}{}{0 em}
  {\indent}



%----------------- fancy headers -------------%

\makeatletter
\patchcmd{\@fancyhead}{\rlap}{\color{grisf}\rlap}{}{}
\patchcmd{\headrule}{\hrule}{\color{grisf}\hrule}{}{}
\patchcmd{\@fancyfoot}{\rlap}{\color{grisf}\rlap}{}{}
\patchcmd{\footrule}{\hrule}{\color{grisf}\hrule}{}{}
\makeatother

                                                                    
\fancyhf{}
\fancyhead[R]{\sffamily\colorb{Report}}
\fancyfoot[R]{\sffamily\small\colorb{\thepage/\pageref{LastPage}}}
\fancyhead[L]{\sffamily\small\colorb{Jean-Baptiste Keck}}
\fancyfoot[L]{\sffamily\small\colorb{M2 MSIAM -- Inverse Methods -- 2014-2015}}
\renewcommand{\headrulewidth}{0.2pt} %0.4
\renewcommand{\footrulewidth}{0.2pt} %0
\addtolength{\headheight}{0.pt}

\fancypagestyle{plain}{
  \fancyhead{}
  \renewcommand{\headrulewidth}{0pt}
  }
     
  %-- macros --%   
  \def\hlinewd#1{%
      \noalign{\ifnum0=`}\fi\hrule \@height #1 %
  \futurelet\reserved@a\@xhline} 

  
  
  %------------------- front page ------------------%
  \title{
      \bsc{TP Inverse Methods}
      \vskip 1cm
      {\colorb\textbf{Lorenz Equations}}
      \vskip 1cm
      \colorb\textit{Report}
  }
\author{%
    Jean-Baptiste \bsc{Keck}
    \vskip 0.5cm
    \bsc{M2 Msiam}
}
%\date{27 janvier 2014}
\makeatletter

\def\maketitle{%
    %\thispagestyle{empty}%
    \begin{flushleft}
        \normalfont\LARGE\par
    \end{flushleft}
    \vskip 3cm
    \begin{center}%
        {\colora\specialrule{.2em}{0em}{0em}}
        \vskip 1cm
        {\Huge \@title}%
        \vskip 1cm
        {\colora\specialrule{.2em}{0em}{0em}}
        \vskip 5cm
        {\Huge \@author\par}%
        \vskip 2cm
        {\Huge \@date\par}%
        \vskip 1cm

    \end{center}%
    \clearpage
}

\renewcommand{\ccite}[1]{\textbf{\cite{#1}}}
\renewcommand{\pd}[2]{\dfrac{\partial #1}{\partial #2}}
\renewcommand{\abs}[1]{\left( #1 \right)}
\renewcommand{\norm}[1]{\lVert #1 \rVert}
\renewcommand{\dx}{\ \mathrm{d}x}
\renewcommand{\dt}{\ \mathrm{d}t}
\renewcommand{\u} {\mathbf{u}}
\renewcommand{\v} {\mathbf{v}}
\renewcommand{\w} {\mathbf{w}}
\renewcommand{\h} {\mathbf{h}}
\renewcommand{\f} {\mathbf{f}}
\renewcommand{\g} {\mathbf{g}}
\renewcommand{\uz}{\mathbf{u_0}}
\renewcommand{\ud}{\mathring{\mathbf{u}}}
\renewcommand{\ut}{\tilde{\mathbf{u}}}
\renewcommand{\utrue}{\mathbf{u}^t}}
\renewcommand{\uobs}{\mathbf{u}^{obs}}}
\renewcommand{\xobs}{x^{obs}}}
\renewcommand{\utz}{\tilde{\mathbf{u}}_0}
\renewcommand{\uh}{\hat{\mathbf{u}}}
\renewcommand{\vh}{\hat{\mathbf{v}}}
\renewcommand{\wh}{\hat{\mathbf{w}}}
\renewcommand{\ub}{\bar{\mathbf{u}}}
\renewcommand{\F}{\mathrm{F}}
\renewcommand{\R}{\mathbb{R}}
\renewcommand{\J}{\mathcal{J}}
\renewcommand{\xt}{\tilde{x}}
\renewcommand{\xh}{\hat{x}}
\renewcommand{\yh}{\hat{y}}
\renewcommand{\zh}{\hat{z}}
\renewcommand{\xb}{\bar{x}}

\newcommand{\bigo}[1]{\ensuremath{\mathop{}\mathopen{}O\mathopen{}\left(#1\right)}}
\newcommand{\smallo}[1]{\ensuremath{\mathop{}\mathopen{}o\mathopen{}\left(#1\right)}}

\definecolor{lightgray}{gray}{0.9}
\renewcommand{\colbox}[1]{\colorbox{lightgray}{$ #1 $}}
%%%%%%%

\begin{document}
\pagestyle{fancy}

\maketitle

%\tableofcontents
\clearpage

\section{Theorical work on the continuous model}

\textbf{Lorentz Equations :}
\vskip 0.2cm

$$
\begin{equation} \label{eq:lorentz}
\left \{
\begin{array}{l c l}
    \mathring{x} & = & 10(y - x)\\
    \mathring{y} & = & 28x - y -xz\\
    \mathring{z} & = & xy - \frac{8}{3}z\\
    \multicolumn{3}{l}{x(0) = x_0,\ y(0) = y_0,\ z(0) = z_0}
\end{array}
\right .
\hskip 0.5cm
\Leftrightarrow 
\hskip 0.5cm
\colbox{
\left \{
\begin{array}{l c l}
\ud&=&\f(\u) \\
\u_0&=&[x_0, y_0, z_0]^T
\end{array}
\right .
}
\end{equation}
$$

with $\u(t) = 
\left [
    \begin{array}{c}
        x(t)&
        y(t)&
        z(t)
    \end{array}
\right ]$
and $\f(\u) =
\left [
    \begin{array}{c}
        10(y - x)\\
        28x - y -xz\\
        xy - \frac{8}{3}z\\
    \end{array}
\right ]
$

\vskip 0.5cm
\subsection{Question 1:} 

\vskip 0.3cm
\noindent Let $\J$ be the following cost function :
\vskip 0.3cm

$\colbox{\J(\u_0) = \dfrac{1}{2}\displaystyle \int_0^T\norm{\u(t)-\uobs(t)}^2 \dt}$ where $\u$ is the solution of (1) with initial condition $\u_0$.

\vskip 0.5cm
\noindent Let $T \in \R^*_+,\ \alpha \in \R^*,\ \u_0 \in \R^3,\ \delta\u_0 \in \R^3$
and
$\ut_0 = \u_0 + \alpha\ \delta\u_0$.
\vskip 0.1cm
\noindent Let $\u$ (resp. $\ut$) be the unique solution of the system (\ref{eq:lorentz}) with $\u_0$ (resp. $\ut_0$) as initial condition.
\vskip 0.1cm
\noindent Let $\ub = \dfrac{\ut - \u}{\alpha}$,  $\uh = \lim\limits_{\alpha \to 0} \ub$
}
and assume 
$\mathrm{K} = \left\{t \in\ ]0,T[\ \ |\ \ \norm{\uh(t)} = +\infty \right\}$
is a measure-zero set.
\vskip 0.5cm

$$
\begin{equation} \label{eq:grad}
    \colbox{
        d\left(\J(\u_0)\right)(\delta \u_0) 
        =\nabla \J(\u_0) \cdot \delta\u_0
        = \lim\limits_{\alpha \to 0}\dfrac{\J(\ut_0) - \J(\u_0)}{\alpha}\ 
    }
\end{equation}
$$

$
\mathcal{J}_\alpha
=\dfrac{\J(\utz) - \J(\uz)}{\alpha}
= \dfrac{1}{2\alpha}\displaystyle \int_0^T \left[ \norm{\ut(t) - \uobs(t)}^2 - \norm{\u(t)-\uobs(t)}^2 \right ] \dt
\vskip 0.2cm
\hskip 3.75cm
=\dfrac{1}{2\alpha}\displaystyle \int_0^T \left[ \norm{\u(t) - \uobs(t) + \alpha\ \ub(t)}^2 - \norm{\u(t)-\uobs(t)}^2 \right ] \dt
\vskip 0.2cm
\hskip 3.75cm
=\dfrac{1}{2\alpha}\displaystyle \int_0^T \left[ \alpha^2\ \norm{\ub(t)}^2 + 2 \langle \u(t) - \uobs(t), \alpha\ \ub(t) \rangle\right] \dt
\vskip 0.2cm
\hskip 3.75cm
=\dfrac{1}{2}\displaystyle \int_0^T \left[ \alpha\ \norm{\ub(t)}^2 + 2 \langle \u(t) - \uobs(t), \ub(t) \rangle\right] \dt
\vskip 0.2cm
\hskip 3.75cm
$

$
\Rightarrow
d\left(\J(\u_0)\right)(\delta \u_0) 
= \lim\limits_{\alpha \to 0} \mathcal{J}_\alpha
= \lim \limits_{\alpha \to 0} \dfrac{1}{2}\displaystyle \int_0^T \left[ \alpha\ \norm{\ub(t)}^2 + 2 \langle \u(t) - \uobs(t), \ub(t) \rangle\right] \dt
\vskip 0.2cm
\hskip 3.2cm
= \lim \limits_{\alpha \to 0} 
\left [ \dfrac{1}{2}\displaystyle  \int_{]0,T[ \backslash K} \alpha\ \underbrace{\norm{\ub(t)}^2}_{\leq M} \dt \right ]
+ \displaystyle \int_0^T \langle \u(t) - \uobs(t), \uh(t) \rangle \dt
\vskip 0.2cm
\hskip 3.2cm
= 0 + \displaystyle \int_0^T \langle \u(t) - \uobs(t), \uh(t) \rangle \dt
$

\vskip 0.2cm
$$
\begin{equation} \label{eq:diff}
\colbox{
d\left(\J(\u_0)\right)(\delta \u_0) 
= \displaystyle \int_0^T \left[ \u(t) - \uobs(t) \right]^T \uh(t) \dt
}
\end{equation}
$$

\noindent\textbf{Tangent model :} 

\vskip 0.2cm
$
\widetilde{(1)} - (1) 
\Leftrightarrow
\left \{
\begin{array}{l c l}
    \mathring{\ut} - \mathring{\u} &=&\f(\ut) - \f(\u)\\
    \ut(0) - \u(0)&=&\ut_0 - \u_0
    \right .
\end{array}
\right.
\vskip 0.2cm
\hskip 1.55cm
\Leftrightarrow
\left \{
\begin{array}{l c l}
    \alpha\ \mathring{\ub} &=& \f(\u + \alpha \ub) - \f(\u)\\
    \alpha \ub(0)&=&\alpha\ \delta \u_0
    \right .
}
\end{array}
\right.
\vskip 0.2cm
\hskip 1.55cm
\Rightarrow
\left \{
\begin{array}{l c l c l c l}
    \mathring{\uh} &=&
    \lim\limits_{\alpha \to 0}  \dfrac{\f(\u + \alpha \ub) - \f(\u)}{\alpha}&=&
    d(\f(\u))(\uh) = J_f(\u)\ \uh\\
    \uh(0)&=&\delta \u_0&&
    \right .
}
\end{array}
$

$$
\begin{equation} \label{eq:tangent}
\colbox{
\left \{
\begin{array}{l c l}
    \pd{\uh}{t} &=&J_f(\u)\ \uh\\
    \uh(0)&=&\delta \u_0
    \right .
\end{array}
}
\textrm{ with }
\colbox{
    J_f(\u) = \pd{\f}{\u}(\u) = 
\left [
    \begin{array}{c c c}
        -10&10&0\\
        28-z&-1&-x\\
        y&x&-\frac{8}{3}
    \end{array}
\right ]
}
\end{equation}
$$

\noindent\textbf{Adjoint model :} 
\vskip 0.2cm

We still have  equation (5):

\vskip 0.5cm
Knowing $\u(t)$ and $\uobs(t)\ \forall t\in [0,T]$, and if we take $\vh$ solution of the following adjoint model:

$$
\begin{equation} \label{eq:adjoint}
\hskip 0.5cm
\colbox{
\left \{
\begin{array}{l c l}
    \mathring{\vh}&=&\g(\vh,t) + \h(t)\\
    \vh(T)&=&\vec{0}
\end{array}
\right .
}
\textrm{ with }
\left \{
\begin{array}{l l l l l}
\g(\vh,t)&=&-J_f^*(\u(t))\ \vh(t)&=&-J_f^T(\u(t))\ \vh(t)\\
\h(t)&=&\u(t) - \uobs(t)
\end{array}
\right .
\end{equation}
$$

By (2), (3) and (5) we get directly $\colbox{\nabla \J(\uz) = - \vh(0)}$.





\vskip 0.5cm
\subsection{Question 2:} 
\vskip 0.3cm
\noindent Let now $\J_x$ be the following cost function :
\vskip 0.3cm

$\colbox{\J_x(\u_0) = \dfrac{1}{2}\displaystyle \int_0^T\abs{x(t)-\xobs(t)}^2 \dt}$ where $x$ is the solution of (1) projected on the x-axis with initial condition $\u_0$.

\vskip 0.5cm
With the same notations and assumptions as before on $x$ :

$$
\begin{equation} \label{eq:grad}
    \colbox{
        d\left(\J_x(\u_0)\right)(\delta \u_0) 
        =\nabla \J_x(\u_0) \cdot \delta\u_0
        = \lim\limits_{\alpha \to 0}\dfrac{\J_x(\ut_0) - \J_x(\u_0)}{\alpha}\ 
    }
\end{equation}
$$

$
\mathcal{J}_x^\alpha
=\dfrac{\J(\utz) - \J(\uz)}{\alpha}
= \dfrac{1}{2\alpha}\displaystyle \int_0^T \left[ \abs{\xt(t) - \xobs(t)}^2 - \abs{x(t)-\xobs(t)}^2 \right ] \dt
\vskip 0.2cm
\hskip 3.75cm
=\dfrac{1}{2\alpha}\displaystyle \int_0^T \left[ \abs{x(t) - \xobs(t) + \alpha\ \xb(t)}^2 - \abs{x(t)-\xobs(t)}^2 \right ] \dt
\vskip 0.2cm
\hskip 3.75cm
=\dfrac{1}{2\alpha}\displaystyle \int_0^T \left[ \alpha^2 \xb(t)^2 + 2\alpha\ [x(t) - \xobs(t)]\ \xb(t) \right] \dt
\vskip 0.2cm
\hskip 3.75cm
=\dfrac{1}{2}\displaystyle \int_0^T \left[ \alpha \xb(t)^2 + 2\ [x(t) - \xobs(t)]\ \xb(t) \right] \dt
\vskip 0.2cm
\hskip 3.75cm
$

$
\Rightarrow
d\left(\J_x(\u_0)\right)(\delta \u_0) 
= \lim\limits_{\alpha \to 0} \mathcal{J}_x^\alpha
= \lim \limits_{\alpha \to 0} \dfrac{1}{2}\displaystyle \int_0^T \left[ \alpha\ \xb(t)^2 + 2\ [x(t) - \xobs(t)]\ \xb(t) \right] \dt
\vskip 0.2cm
\hskip 3.2cm
= \lim \limits_{\alpha \to 0} 
\left [ \dfrac{1}{2}\displaystyle  \int_{]0,T[ \backslash K_x} \alpha\ \underbrace{\xb(t)^2}_{\leq M} \dt \right ]
+ \displaystyle \int_0^T [x(t) - \xobs(t)]\ \xh(t) \dt
\vskip 0.2cm
\hskip 3.2cm
= 0 + \displaystyle \int_0^T [x(t) - \xobs(t)]\ \xh(t) \dt
$

\vskip 0.2cm
$$
\begin{equation} \label{eq:diff}
\colbox{
d\left(\J_x(\u_0)\right)(\delta \u_0) 
= \displaystyle \int_0^T [x(t) - \xobs(t)]\ \xh(t) \dt
}
\end{equation}
$$

\noindent\textbf{Tangent model :} 

\vskip 0.2cm
We still have the same tangent model as before (4), and we can project the equations on $x$, $y$ and $z$ to obtain :

$$
\begin{equation} \label{eq:tangent}
    \colbox{
        \left \{
            \begin{array}{l c l}
                \pd{\xh}{t} &=&\pd{f_x}{\u}(\u)\cdot \uh\\
                \\
                \pd{\yh}{t} &=&\pd{f_y}{\u}(\u)\cdot \uh\\
                \\
                \pd{\zh}{t} &=&\pd{f_z}{\u}(\u)\cdot \uh\\
                \\
                \uh(0)&=&\delta \u_0
            \end{array}
        \right .
        \textrm{ with }
        \left\{
                \begin{array}{c c c c l c c c l}
                    \pd{f_x}{\u}(\u)&=&\nabla f_x(\u)&=& [-10 & 10 & 0]^T\\
                    \\
                    \pd{f_y}{\u}(\u)&=&\nabla f_y(\u)&=& [28-z & -1 & -x]^T\\
                    \\
                    \pd{f_z}{\u}(\u)&=&\nabla f_z(\u)&=& [y & x & -\frac{8}{3}]^T\\
                \end{array}
        \right .
    }
\end{equation}
$$

\noindent\textbf{Adjoint model :} 
\vskip 0.2cm

If we take a new function $\wh$ and we integrate the scalar product by part, we still get (5) :

$$
\colbox{
\displaystyle \int_0^T 
\left[ 
    \wh'(t) + J_f^*(\u(t)) \wh(t) 
\right ]
\cdot \uh(t)
\right] \dt
=
-  \wh(0) \cdot \delta\uz
}
$$

\vskip 0.5cm
Knowing $\u(t)$ and $\uobs(t)\ \forall t\in [0,T]$, and if we take $\wh$ solution of the following adjoint model:

$$
\begin{equation} \label{eq:adjoint}
\hskip 0.5cm
\colbox{
\left \{
\begin{array}{l c l}
    \mathring{\wh}&=&\g(\wh,t) + \h^x(t)\\
    \wh(T)&=&\vec{0}
\end{array}
\right .
}
\textrm{ with }
\left \{
\begin{array}{l l l l l}
    \g(\wh,t)&=&-J_f^*(\u(t))\ \wh(t)&=&-J_f^T(\u(t))\ \wh(t)\\
    \h^x(t)&=&\left[ x(t) - \xobs(t), 0, 0 \right]^T
\end{array}
\right .
\end{equation}
$$

The only thing that changed in the adjoint model is the expression of $h(t)$, the new adjoint model does no more take into acount the distance between $y$ and $y^{obs}$ or $z$ and $z^{obs}$.
With this new adjoint model, we still have the same expression for the gradient : $\colbox{\nabla \J_x(\uz) = - \wh(0)}$.

\section{Discrete model}
\section{Data assimilation experiments}

\end{document}




 
 
 
 
 
 
 
 
 
 
 
 
 
