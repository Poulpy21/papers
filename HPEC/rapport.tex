\documentclass[11pt,a4paper]{article} 

\usepackage[T1]{fontenc}
\usepackage[utf8]{inputenc}
\usepackage{lmodern}
\usepackage[english,frenchb]{babel}

\usepackage{amsthm}
\usepackage{float}
\usepackage{lmodern}%pour un meilleur rendu des polices
\usepackage{verbatim}%du texte non interprt
\usepackage[cmex10]{amsmath}
\usepackage{amssymb}%maths
\usepackage{xspace}
\usepackage[dvipsnames,svgnames,table]{xcolor}
\usepackage{listings}
\usepackage{fancyhdr}
\usepackage{etoolbox}
\usepackage{titlesec}
\usepackage{titletoc}
\usepackage{lastpage}
\usepackage[bookmarks=true,bookmarksnumbered=true]{hyperref}
\usepackage{ctable} % for \specialrule command
\usepackage{cite}
\usepackage{algorithm2e}
\usepackage{alltt}
\usepackage{array}
\usepackage{mdwmath}
\usepackage{mdwtab}
\usepackage{eqparbox}
\usepackage[caption=false,font=normalsize,labelfont=sf,textfont=sf]{subfig}
\usepackage{dblfloatfix}
\usepackage{url}
\usepackage{tipa}
\usepackage{stmaryrd}

%\usepackage{natbib}
%\usepackage[pdftex]{graphicx}
%\usepackage{framed}
%\usepackage[usenames]{color}

\graphicspath{{img/}}
\DeclareGraphicsExtensions{.pdf,.jpeg,.jpg,.png}

%% taille du papier
\textwidth 16 true cm
\textheight 24 true cm
\addtolength{\hoffset}{-1.5cm}
\addtolength{\voffset}{-1.5cm}

%-------- couleurs
\definecolor{grisf}{rgb}{.47,.47,.47} % barre de droite gris fonce
\newcommand{\colorc}{\color{MidnightBlue}}
\newcommand{\colorb}{\color{NavyBlue}}
\newcommand{\colora}{\color{Cerulean}}

%----------- sections et TOC
% chapitres
\titleformat{\chapter}[display]
  {\normalfont\sffamily\bfseries\huge\colora\centering}{\thechapter}{1ex}
  {{\titlerule[1pt]}\vspace{1.3ex}}[\vspace{1ex}{{\titlerule[1pt]}}]
  
% chapitres etoiles  
\titleformat{name=\chapter,numberless}[display]
  {\normalfont\sffamily\bfseries\LARGE\colora\centering}{}{1ex}
  {{\titlerule[1pt]}\vspace{1.3ex}}[\vspace{1ex}{\titlerule[1pt]}\vspace{2ex}]
  
% sections  
\titleformat{\section}[hang]{\Large\normalfont\sffamily\bfseries\colora}{{\thesection\, }}{0 em}
  {}[{\titlerule[1pt]}\vspace{1ex}]

  
% sous section, sous sous sec, paragraphes  
\titleformat{\subsection}[hang]{\Large\normalfont\sffamily\bfseries\colorc}{{\thesubsection\, }}{0 em}
  {}[{\titlerule}\vspace{.7ex}]
\titleformat{\subsubsection}[hang]{\normalfont\sffamily\bfseries\large}{{\thesubsubsection\, }}{0 em}
  {}[{\color{grisf}\titlerule}\vspace{3pt}]
\titleformat{\paragraph}[runin]{\normalfont\sffamily\bfseries\colorb}{}{0 em}
  {\indent}



%----------------- fancy headers -------------%

\makeatletter
\patchcmd{\@fancyhead}{\rlap}{\color{grisf}\rlap}{}{}
\patchcmd{\headrule}{\hrule}{\color{grisf}\hrule}{}{}
\patchcmd{\@fancyfoot}{\rlap}{\color{grisf}\rlap}{}{}
\patchcmd{\footrule}{\hrule}{\color{grisf}\hrule}{}{}
\makeatother

                                                                    
\fancyhf{}
\fancyhead[R]{\sffamily Article summary }
\fancyfoot[R]{\sffamily\small{\thepage/\pageref{LastPage}}}
\fancyhead[L]{\sffamily\small{Jean-Baptiste Keck}}
\fancyfoot[L]{\sffamily\small{M2 MSIAM -- High Performance Exact Computing -- 2014-2015}}
\renewcommand{\headrulewidth}{0.2pt} %0.4
\renewcommand{\footrulewidth}{0.2pt} %0
\addtolength{\headheight}{0.pt}

\fancypagestyle{plain}{
  \fancyhead{}
  \renewcommand{\headrulewidth}{0pt}
  }
     
  %-- macros --%   
  \def\hlinewd#1{%
      \noalign{\ifnum0=`}\fi\hrule \@height #1 %
  \futurelet\reserved@a\@xhline} 

  
  
  %------------------- front page ------------------%
  \title{
      \bsc{Paper Review}
      \vskip 1cm
      {\colorb\textbf{How to Compute Fast a Function and All Its Derivatives}}
      \vskip 1cm
      {\colorc\textit{A variation on the theorem of Bauer-Strassen}}
  }
\author{%
    Jean-Baptiste \bsc{Keck}
    \vskip 0.5cm
    \bsc{M2 Msiam}
}
%\date{27 janvier 2014}
\makeatletter

\def\maketitle{%
    \thispagestyle{empty}%
    \begin{flushleft}
        \normalfont\LARGE\par
    \end{flushleft}
    \vskip 3cm
    \begin{center}%
        {\colora\specialrule{.2em}{0em}{0em}}
        \vskip 1cm
        {\Huge \@title}%
        \vskip 1cm
        {\colora\specialrule{.2em}{0em}{0em}}
        \vskip 5cm
        {\Huge \@author\par}%
        \vskip 2cm
        {\Huge \@date\par}%
        \vskip 1cm

    \end{center}%
    \clearpage
}

\newcommand{\ccite}[1]{\textbf{\cite{#1}}}
\newcommand{\pd}[2]{\dfrac{\partial #1}{\partial #2}}
%%%%%%%

\begin{document}
\pagestyle{fancy}

\maketitle

%\tableofcontents
%\clearpage

\section{Introduction}

This paper is about an alternative algorithmic proof of the Baur-Strassen result given and proved in \ccite{1}, showing that the complexity of the evaluation of a rational function of several variables and all its derivatives is bounded above by three times the complexity of the evaluation of the function itself.

\section{Notations}

The notation is mainly the one used in the original paper \cite{0} and \cite{1} with some minor changes.

\vskip 0.5cm
$\mathbb{K}$ is an infinite field.\\
\indent$F$ is a rational function of $n$ variables $x_1,x_2,...,x_n$.\\
\indent$\widetilde{F}$ is a rational function of $n+1$ variables $x_1,x_2,...,x_n,y$.
\vskip 0.3cm
\indent The set of partial derivatives of F is denoted $F' = \left \{ \pd{F}{x_1},\pd{F}{x_2}, ... ,\pd{F}{x_n} \right \}.
\vskip 0.3cm
Let \textbullseye\, be an arithmetical operation and $x = (x_1,x_2,...,x_n)$.\\
$$
\left(\textrm{a \textbullseye\, b is an \textbf{essential operation}}\right)
\Leftrightarrow
\[
   \left \{
   \begin{array}{l}
       a = f(x)\textrm{ and }b = g(x) \\
       \textrm{\textbf{or }}a \in \mathbb{K}, b = g(x)\textrm{ and \textbullseye\, is a division.}
   \end{array}
   \right .
\]
$$

Let \u{A} be an algorithm computing F from $x_1,x_2,...,x_n$ and $\mathbb{K}$.
\vskip 0.3cm
$$
\exists\,\, u \in \mathbb{N} \,\,\, \u{A} = \left\{ g_1,g_2,...,g_u \right\}
\textrm{ with }
\[
   \left \{
   \begin{array}{l}
       g_k \in \left\{x_1,x_2,...,x_n\right\} \cup \mathbb{K}\\
       \textrm{\textbf{or }} g_k = g_{k_1} \textrm{ \textbullseye\, } g_{k_2}\textrm{ with }k_1,k_2 < k
   \end{array}
   \right .
\]
$$

Finally let :

\begin{itemize}
    \item $s(F)$ be the number of \textbf{essential} multiplications and divisions in $\u{A}$.
    \item $m(F)$ be the \textbf{total} number of multiplications and divisions in $\u{A}$.
    \item $T(F)$ the total number of \textbf{essential operations} in $\u{A}$.
    \item $\Theta(F)$ the total number of operations in $\u{A}$.
\end{itemize}

\section{Baur-Strassen's Theorem}
    From each algorithm $\u{A}$ computing $F$ one can derive and algorithm $\mathring{A}$ computing $F$ and $F'$ such that :

$$
    (P) \Leftrightarrow
    \[
   \left \{
   \begin{array}{l c l}
       s(F,F') &\leq & \mathbf{3} \,s(F)\\
       m(F,F') &\leq & \mathbf{3} \,m(F)\\
       T(F,F') &\leq & \mathbf{5} \,T(F)\\
       \Theta(F,F') &\leq & \mathbf{5} \,\Theta(F)
       
   \end{array}
   \right .
\]
$$

\vskip 0.5cm
\textbf{Those inequalities are independent of the number of variables n.}

\section{Proof overview}

The proof is made by \textbf{induction on the length of the algorithm}. Let $\u{A}_u$ be an algorithm of length u computing a rational function $F$ and $g_1$ be the result of the first operation of $\u{A}_u$.
\vskip 0.2cm
Define $\widetilde{F}$ a function of $n+1$ variables such that $F(x_1,x_2,...,x_n)=\widetilde{F}(x_1,x_2,...,x_n,\mathbf{y})$ with $y = g_1(x_1,x_2,...,x_n)$.
\vskip 0.2cm
$\u{A}_u}$ induces an algorithm $\u{A}_{u-1}$ which computes $\widetilde{F}$ from $x_1,x_2,...,x_n,g_1$ and $\mathbb{K}$ in one less operation ($\u{A}_{u-1}$ is of length $u-1$). \textbf{By induction hypothesis, $\widetilde{F}$ satisfies (P).}
\vskip 0.5cm
We have $\forall h \in \llbracket 1,n \rrbracket \,\,\, \pd{F}{x_h} = \pd{\widetilde{F}}{x_h} + \pd{\widetilde{F}}{y}\cdot\pd{g_1}{x_h}\ \ (1).$

\vskip 0.5cm
\textbf{The idea is then to examine all six possible cases :}
\vskip 0.3cm

1. Example case $\mathbf{g_1 = \textcolor{red}{c \cdot x_i}}$ where $c \in \mathbb{K}$ and $i \in \llbracket 1,n \rrbracket$ : 
$$
    (1) \Rightarrow 
    \[
   \left \{
   \begin{array}{l r}
   \pd{F}{x_h}(x_1,...,x_n) = \textcolor{purple}{\pd{\widetilde{F}}{x_h}(x_1,...,x_n) \,\mathbf{+}}\, \textcolor{blue}{c \cdot \pd{\widetilde{F}}{y}(x_1,...,x_n)} & \textrm{ if }h = i\\ \\
       \pd{F}{x_h}(x_1,...,x_n) = \pd{\widetilde{F}}{x_h}(x_1,...,x_n) & \textrm{ if } h \neq i
   \end{array}
   \right .
\]
$$

\vskip 0.5cm
$
\textrm{In this case we have :}
\[
   \left \{
   \begin{array}{l c l l l l l l l}
   s(F,F') & = & s(\widetilde{F}},\widetilde{F}') &&\\
   m(F,F') &\leq & m(\widetilde{F}},\widetilde{F}')&+&\textcolor{blue}{1}&+&\textcolor{red}{1}\\
   T(F,F') &\leq & T(\widetilde{F}},\widetilde{F}')&+&\textcolor{purple}{1}\\
       \Theta(F,F') &\leq & \Theta(\widetilde{F},\widetilde{F}')&+&\textcolor{purple}{1}&+&\textcolor{blue}{1}&+&\textcolor{red}{1}
   \end{array}
   \right .
\]
$

\vskip 0.5cm
2. Case $\mathbf{g_1 = \textcolor{red}{x_i \cdot x_j}}$ where $i,j \in \llbracket 1,n \rrbracket$.


$$
    (1) \Rightarrow 
    \[
   \left \{
   \begin{array}{l r}
   \pd{F}{x_h}(x_1,...,x_n) = \textcolor{purple}{\pd{\widetilde{F}}{x_h}(x_1,...,x_n) \,\mathbf{+}}\, \textcolor{blue}{x_j \cdot \pd{\widetilde{F}}{y}(x_1,...,x_n)} & \textrm{ if }h = i\\ \\
   \pd{F}{x_h}(x_1,...,x_n) = \textcolor{purple}{\pd{\widetilde{F}}{x_h}(x_1,...,x_n) \,\mathbf{+}}\, \textcolor{blue}{x_i \cdot \pd{\widetilde{F}}{y}(x_1,...,x_n)} & \textrm{ if }h = j\\ \\
       \pd{F}{x_h}(x_1,...,x_n) = \pd{\widetilde{F}}{x_h}(x_1,...,x_n) & \textrm{ if } h \notin \{i,j\}
   \end{array}
   \right .
\]
$$

\vskip 0.5cm
$
\textrm{In this case we have :}
\[
   \left \{
   \begin{array}{l c l l l l l l l}
   s(F,F') &\leq & s(\widetilde{F}},\widetilde{F}')&+&\textcolor{blue}{2}&+&\textcolor{red}{1}\\
   m(F,F') &\leq & m(\widetilde{F}},\widetilde{F}')&+&\textcolor{blue}{2}&+&\textcolor{red}{1}\\
       T(F,F') &\leq & T(\widetilde{F},\widetilde{F}')&+&\textcolor{purple}{2}&+&\textcolor{blue}{2}&+&\textcolor{red}{1}&
       \Theta(F,F') &\leq & \Theta(\widetilde{F},\widetilde{F}')&+&\textcolor{purple}{2}&+&\textcolor{blue}{2}&+&\textcolor{red}{1}
   \end{array}
   \right .
\]
$

\clearpage
\textbf{We do exactly the same for the 4 other cases :}
\vskip 0.5cm
    3. Case $\mathbf{g_1 = c/x_i}$ where c \in \mathbb{K}$ and $i \in \llbracket 1,n \rrbracket$.
\vskip 0.1cm
    4. Case $\mathbf{g_1 = x_i/x_j}$ where $i,j \in \llbracket 1,n \rrbracket$.
\vskip 0.1cm
    5. Case $\mathbf{g_1 = x_i+d}$ where $d \in \mathbb{K}$ and $i \in \llbracket 1,n \rrbracket$.
\vskip 0.1cm
    6. Case $\mathbf{g_1 = x_i+x_j}$ where $i,j \in \llbracket 1,n \rrbracket$.

\vskip 0.5cm
Putting everything together and using (P) for $\widetilde{F}$, we get at most the inequalities:
$$
\[
   \left \{
   \begin{array}{l c l l l c l l l c l}
   s(F,F') &\leq & s(\widetilde{F}},\widetilde{F}')&+&3&\leq&3\,s(\widetilde{F})&+&3&\leq&3\,s(F)\\
   m(F,F') &\leq & m(\widetilde{F}},\widetilde{F}')&+&3&\leq&3\,m(\widetilde{F})&+&3&\leq&3\,m(F)\\
   T(F,F') &\leq & T(\widetilde{F}},\widetilde{F}')&+&5&\leq&5\,T(\widetilde{F})&+&5&\leq&5\,T(F)\\
       \Theta(F,F') &\leq & \Theta(\widetilde{F},\widetilde{F}')&+&5&\leq&5\,\Theta(\widetilde{F})&+&5&=&5\,\Theta(F)
   \end{array}
   \right .
\] \Rightarrow \textrm{(P) for F, which proves the theorem since the beginning of the induction is trivial.}
$$

\vskip 0.5cm

Those inequalities are independent of the number of variables n since we just added a bounded number of new operations.

\section{Conclusion}

From this new proof of the theorem we can deduce a constructive recursive backward way to build an algorithm for the computation of all partial derivatives of $F$ at a given point $x$ from any algorithm computing $F$ at this point.



\bibliographystyle{plain}
\bibliography{master}

\end{document}




 
 
 
 
 
 
 
 
 
 
 
 
 
